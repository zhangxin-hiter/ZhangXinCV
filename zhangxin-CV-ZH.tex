%%%%%%%%%%%%%%%%%%%%%%%%%%%%%%%%%%%%%%%%%
% "ModernCV" CV and Cover Letter
% LaTeX Template
% Version 1.1 (9/12/12)
%
% This template has been downloaded from:
% http://www.LaTeXTemplates.com
%
% Original author:
% Xavier Danaux (xdanaux@gmail.com)
%
% License:
% CC BY-NC-SA 3.0 (http://creativecommons.org/licenses/by-nc-sa/3.0/)
%
% Important note:
% This template requires the moderncv.cls and .sty files to be in the same
% directory as this .tex file. These files provide the resume style and themes
% used for structuring the document.
%
%%%%%%%%%%%%%%%%%%%%%%%%%%%%%%%%%%%%%%%%%
% 最后更新:2014年10月11日
%----------------------------------------------------------------------------------------
%   PACKAGES AND OTHER DOCUMENT CONFIGURATIONS
%----------------------------------------------------------------------------------------

\documentclass[10pt,a4paper,sans]{moderncv} % Font sizes: 10, 11, or 12; paper sizes: a4paper, letterpaper, a5paper, legalpaper, executivepaper or landscape; font families: sans or roman
% moderncv version 1.5.1 (29 Apr 2013)


\moderncvstyle{banking} % CV theme - options include: 'casual' (default), 'classic', 'oldstyle' and 'banking'
\moderncvcolor{black} % CV color - options include: 'blue' (default), 'orange', 'green', 'red', 'purple', 'grey' and 'black'

\usepackage[noindent]{ctex} %中文支持
\setCJKmainfont{SimSun}

%\usepackage{lipsum} % Used for inserting dummy 'Lorem ipsum' text into the template

\usepackage[top=1cm,bottom=1cm,left=2cm,right=2cm]{geometry} % Reduce document margins
%\setlength{\hintscolumnwidth}{3cm} % Uncomment to change the width of the dates column
%\setlength{\makecvtitlenamewidth}{10cm} % For the 'classic' style, uncomment to adjust the width of the space allocated to your name

%----------------------------------------------------------------------------------------
%   NAME AND CONTACT INFORMATION SECTION
%----------------------------------------------------------------------------------------
\name{张}{昕}
% All information in this block is optional, comment out any lines you don't need
\title{个人简历}
\address{哈尔滨工业大学(深圳)}{南山区, 深圳市 518052}
\phone[mobile]{(+86)18870619179}
\email{zx2567617517@foxmail.com}
%\homepage{www.example.com}
%\social[twitter]{username}
%\social[github]{username}
%\extrainfo{additional information}
%\photo[70pt][0.4pt]{photo} % The first bracket is the picture height, the second is the thickness of the frame around the picture (0pt for no frame)
%\quote{"A witty and playful quotation" - John Smith}

%----------------------------------------------------------------------------------------

\begin{document}

\makecvtitle % Print the CV title

%----------------------------------------------------------------------------------------
%   POSITION APPLIED(CAREER OBJECTIVE)
%----------------------------------------------------------------------------------------
%\section{求职意向}
%%\subsection{求职意向}
%\cventry{期望月薪:  面议}{应聘职位:初级硬件工程师}{}{}{}{}

%----------------------------------------------------------------------------------------
%   EDUCATION SECTION
%----------------------------------------------------------------------------------------

\section{教育背景}

\cventry{2022---2026}{通信工程}{哈尔滨工业大学(深圳)}{本科}{}{}{\textit{GPA: 89/100(前15\%)}  % Arguments not required can be left empty

%----------------------------------------------------------------------------------------
%   WORK EXPERIENCE SECTION
%----------------------------------------------------------------------------------------

\section{项目经历}
%------------------------------------------------
\subsection{华大九天杯深圳市电子设计邀请赛}

\cventry{2025.04---2025.04}{项目成员}{视觉导航智能小车}{省市级}{}{
\begin{itemize}
\setlength{\itemindent}{2em}
%   \item 设计测试方法、搭建测试平台,编写自动化脚本,完成FFmpeg/X264 编码速度测试
    \item 小车通过摄像头识别道路红线,完成道路寻线,并利用神经网络识别道路标识,根据标识完成停车,左转右转等动作
    \item 个人负责:视觉模块K230与主控芯片MSPM0G3507板间通信数据传输,ADC按键,蜂鸣器,小车与上位机无线通信部分
\end{itemize}
}
%%------------------------------------------------
%\subsection{首都师范大学``本科生毕业论文(设计)"}
%
%\cventry{2012--2013}{毕设小组负责人}{基于DDS 技术的FM信号调制的设计及其FPGA 实现}{}{}{
%\begin{itemize}
%%  \item 负责基于直接数字频率合成(DDS) 技术设计和实现频率调制功能
%   \item 基于Altera 公司的Cyclone II 系列FPGA 芯片,修改开发平台示例程序,完成系统编程
%    \item 参与定制小型自制实验平台(基于DE2开发板)的芯片选型和硬件设计工作
%\end{itemize}
%}
%------------------------------------------------
\subsection{个人项目}

\cventry{2025.01---2025.01}{项目负责人}{智能蓝牙避障小车}{校级}{}{
\begin{itemize}
\setlength{\itemindent}{2em}
%   \item 负责解决FIR的IP核的设计,利用Audio ADC/DAC引脚来设计音频输入
    \item 实现小车自动避障功能,并通过蓝牙返回到障碍物的距离
    \item 负责主控芯片为stm32f103c8t6,通过定时器的输出比较功能输出两路pwm分别控制两边电机,蓝牙模块通过串口与 主控芯片通信,返回障碍物距离至手机,超声波通过测量小车前左右障碍物距离并将其返回主控,控制小车运动逻辑
\end{itemize}
}

%------------------------------------------------
%------------------------------------------------
\subsection{2024全国电子设计大赛H题(练习)}

\cventry{2025.01---2025.01}{项目成员}{红外巡线小车}{校级}{}{
\begin{itemize}
\setlength{\itemindent}{2em}
\item 实现小车按指定路线寻线
\item 本人负责mpu6050六轴传感器部分。采用I2C通信协议,实现传感器与主控芯片mspm0通信,通过传感器按 照一定频率测量当前各个方向的角速度,并积分获得当前偏航角翻滚角以及俯仰角,再通过测量各个方向的加速度测得角度,与前 者进行数据融合获得小车的偏航角翻滚角以及俯仰角
%\item 论文《面向多核处理器系统的Cache 感知调度算法》发表在中文核心期刊《小型微型计算机系统》
\end{itemize}
}
%------------------------------------------------
%------------------------------------------------
%\subsection{首都师范大学``实验室开放基金"项目}

%\cventry{2011.04---2012.04}{项目负责人}{网络工程创新实验设计——基于Hadoop的海量数据应用研究}{校级}{}{
%\begin{itemize}
%\setlength{\itemindent}{2em}
%\item 带领小组对Hadoop大规模数据排序算法TeraSort进行分析并作为基准测试程序进行应用
%\item 项目被评为首都师范大学``实验室开放基金"优秀成果一等奖(学院唯一一个被评为优秀的项目)
%\end{itemize}
%}
%------------------------------------------------
%\subsection{``国家大学生创新性实验计划"项目}

%\cventry{2010.04---2011.04}{项目核心成员}{Unix/Linux环境下路由管理转换接口设计与实现}{国家级}{}{
%\begin{itemize}
%\setlength{\itemindent}{2em}
%\item 利用脚本和程序来高效实现路由管理转换接口,实现软路由功能 %自学Linux 操作系统各种工具和服务配置,
%\item 查阅大量资料,提前学习网络原理及网络工程相关内容,实现软路由功能(基于GNU Zebra)
%\item 项目通过学校验收,团队合作经验和专业技能得到增强
%\end{itemize}
%}

%------------------------------------------------
%%------------------------------------------------
%\subsection{首都师范大学学位论文\LaTeX 模板开发}
%\cventry{2013}{项目负责人}{首都师范大学学位论文(本科生、硕博)\LaTeX 模板}{开源项目}{}{
%\begin{itemize}
%   \item 开发和维护首都师范大学本科生,硕士生、博士生学位论文\LaTeX 模板
%\end{itemize}
%}

%----------------------------------------------------------------------------------------
%   AWARDS SECTION
%----------------------------------------------------------------------------------------

\section{荣誉奖励}
\cvitem{2023,2024}{2次获得学校学业奖学金}
\cvitem{2023,2024}{两次获得全国大学生数学竞赛黑龙江赛区一等奖}
\cvitem{2025.04}{华大九天杯深圳市电子设计邀请赛三等奖}
\cvitem{2024.10}{哈尔滨工业大学优秀学生荣誉}
%\cvitem{2012.12}{作为大学生创新性实验优秀学生赴韩国校外访学一周(学院仅1个名额)}
%\cvitem{2011.12}{第六届全国信息技术应用水平大赛比赛安卓应用开发团体赛三等奖~团队组长}
%\cvitem{2011.10}{北京市大学生计算机应用大赛移动终端应用创意与程序设计二等奖~团队组长}
%\cvitem{2010,2011,2012}{3次获得国家励志奖学金(每年奖励综合测评排名前5\%$\sim$7\%不等)}
\section{专业技能}
\cvitem{}{熟练使用c语言,能够使用python编程}
\cvitem{}{熟悉stm32,msp系列单片机开发,熟悉i2c,uart,spi等通信协议}
\cvitem{}{熟悉RTOS,了解Linux开发}
\cvitem{}{熟悉multisim,cadence,virtuoso等eda软件}
\cvitem{}{熟悉ad,立创eda等pcb绘制软件}
\cvitem{}{熟悉git,makefile等工具 }
\end{document}
